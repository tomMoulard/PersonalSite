\documentclass[10pt,a4paper,sans]{moderncv}
\moderncvstyle{classic}
% possible styles : 'casual' (by default),
% 'classic', 'oldstyle' and 'banking'
\moderncvcolor{orange}
% possible colors : 'blue' (by default),
% 'orange', 'green', 'red', 'purple', 'grey' and 'black'
\usepackage[top=1.0cm, bottom=1.0cm, left=1.6cm, right=1.6cm]{geometry}
\usepackage{helvet}
\usepackage[utf8]{inputenc}

% Language
\newcommand\en[1]{}
\newcommand\fr[1]{#1}
\newcommand\full[1]{}

% personal data
\name{Tom}{Moulard}
\title{Software Developer}
\email{tom@moulard.org}
\homepage{tom.moulard.org}
\address{\fr{2 rue Guébriant}\en{2, street guébriant}}{75020 Paris}{\en{France}}
\phone[mobile]{\href{tel:+33-6-51-48-72-60} {+33 6 51 48 72 60}}
\social[linkedin]{tommoulard}
\social[github]{tommoulard}
\extrainfo
{
    \fr{
        Permis B - voiture\\
        \full{Proactif en télétravail\\}
        24 ans
    }
    \en{25 years old\full{ - Experienced in remote work}}
}
\fr{\photo[64pt][0.4pt]{tom_v.jpeg}}
\full{\quote{"Our virtues and our failings are inseparable, like force and
matter. When they separate, man is no more" Nikola Tesla}}

\begin{document}
\makecvtitle
\vspace*{-10mm}

\section{\fr{Objectifs}\en{Objectives}}
\fr{
    Actuellement en dernière année d'école d'ingénieur en informatique à
    l'EPITA.
    Je cherche un CDI pour valoriser mon parcours ingénieur.
    Je serai disponible a partir de Septembre 2021.
    \full{
        Je cherche à appliquer mes compétences analytiques aux défis proposés
        par un stage dans votre entreprise.
        Mes capacités en Algorithmique et en recherche permettront à votre
        organisation d'atteindre ses objectifs.
    }
}
\en{
    Seeking to apply my analytical skills to the challenges posed by a
    job at your company.
    Poses proven algorithmic and research abilities that will aid your company
    in meeting its milestones.
}

\section{\fr{\'Education}\en{Education}}
% \cventry{years}{degree}{institution}{location}{grade}{description}
\cventry{2014-date}{EPITA}
{\fr{\'Ecole d'ingénieur en informatique}\en{Computer Engineering}}
{Kremlin-Bicêtre - FR}
{\fr{Membre Prologin et Assistant TCOM}\en{Member of the Prologin Association and TCOM assistant}}{}

\cventry{\fr{Jan}\en{Jun} 2021}
{\href{https://aws.amazon.com/fr/certification/certified-cloud-practitioner/}{Certification AWS Cloud Practitioner}}
{AWS}{\fr{en ligne}\en{online} - \href{https://www.youracclaim.com/badges/4ab6148a-29a4-4b70-82c2-60c8b5c0c473}{v\fr{\'}erification}}
{788/1000}
{}

\cventry{D\fr{\'}ec 2020}
{Certification HCIA Routing \& switching}
{Huawei}{\fr{en ligne}\en{online} - \href{https://e.huawei.com/en/talent/\#/cert/certificate-verification}{v\fr{\'}erification}}
{742/1000}
{}

\full{\cventry{2020}{Huawei}
{\fr{Formation des les équipements et sur eNSP}\en{Training about equipments and eNSP}}
{Visio conference}
{}{}}

\cventry{2014-2019}{Université Paris-Est Créteil}
{\fr{Université d'informatique}\en{Computer Engineering focused School}}
{Créteil - FR}
{\fr{Licence d'informatique}}{}

\full{\cventry{2018}{Google}{Q{\&}A Meeting with an Google Software Engineer.}
{FR}{}{}}

\cventry{2017}{Beijing Jiaotong University}
{\fr{6 mois d'étude informatique}\en{6 month studying IT engineering}}
{\fr{Pékin}\en{Beijing} - ZH}{}{}

\full{\cventry{2014}{BNSSA et PSE1}{\fr{Surveillant de baignade}\en{lifeguard}}{}{}{}}

\cventry{2014}
{\fr{Baccalauréat}\en{High School diploma}}
{\fr{Série Scientifique}\en{Scientific section}}{}
{}{}

\section{Expériences Professionnelles}
% \cventry{years}{position}{company}{location}{length}{description}
\cventry{F\fr{év}\en{eb} 2021-date}{\fr{Développeur}\en{Developer}}{Traefik Labs}{\fr{Télétravail}\en{Remote}}
{}{
    \fr{
        Mainteneur du projet Open Source \href{https://github.com/traefik/traefik}{Traefik Proxy} (Stage de Février à Juillet).
    }
    \en{
        Maintainer of the \href{https://github.com/traefik/traefik}{Traefik Proxy} Open Source Project(intern from February to July).
    }
}
\cventry{2019-2021}{\fr{Co-fondateur et CTO}\en{Co-Funder and CTO}}{Eiko}{FR}
{}{
    \fr{
        Création d'une application web progressive(PWA) en: Go, GCP et Algolia;
        HTML, CSS, JS.
    }
    \en{
        Creation of a Progressive Web App using GO, GCP, and Algolia; HTML, CSS, JS.
    }
}

\cventry{2018}{Full Stack}{Zeodine}{FR}{4 \fr{mois}\en{month}}
{
    \fr{
        Création d'un site internet utilisant: Go, CI/CD; HTML, CSS, JS et P5.js.
    }
    \en{
        Creating a web site using: Go; HTML; CSS; JS and P5.js.
        And using: Unit Testing; Automatic Deployment.
    }
}

\cventry{2017}{\fr{Développeur Python}\en{Python Developer}}
{\fr{Ministère de l'environnement Chinois}}{ZH}{2 \fr{mois}\en{month}}
{
    \fr{
        Natural Language Processing en Python et management de base de donnée
        graphique (Neo4j).
    }
    \en{
        Natural Language Processing in Python and graph database management
        (Neo4j).
    }
}

\cventry{2017}{\fr{Stage}\en{Internship}}{Orange}{FR}{1 \fr{mois}\en{month}}
{
    \fr{
        Management de firewall pfsense dans un Cloud OpenStack
        (Haute disponibilité, tunneling, tri).
    }
    \en{
        pfsense firewall managing inside an OpenStack cloud
        (High Availability, tunneling, sorting).
    }
}

\full{\cventry{2015}{Sauveteur}{Abyssea}{FR}{1 mois}{}}

\section{\fr{Projets}\en{Projects}}
% \cventry{years}{project name}{language}{link}{length}{description}

\cventry{2020}{\fr{Première place au Hackaethon Traefik}\en{First place at the Traefik Hackaethon}}{Golang}
{\href{https://twitter.com/traefik/status/1320768681119490056}{\fr{lien}\en{link}}}{3 \fr{jours}\en{days}}{
    \fr{Première place pour la réalisation de deux plugins \full{middleware }pour le
    proxy \href{https://traefik.io/}{Traefik}:
    \href{https://github.com/tomMoulard/fail2ban}{Fail2Ban} et
    \href{https://github.com/tomMoulard/htransformation}{Header Transformation}.}
    \en{First place awarded for the creation of two \full{middleware }plugins for the
    \href{https://traefik.io/}{Traefik} proxy:
    \href{https://github.com/tomMoulard/fail2ban}{Fail2Ban} and
    \href{https://github.com/tomMoulard/htransformation}{Header Transformation}.}
}

\cventry{2020}{\fr{Projet de transformation SI}\en{IS Transformation project}}{}
{}{\textit{1 an}}
{
    \fr{Projet de fusion de SI de trois compagnies aérienne. Étude de
    l’existant, vision globale et stratégique déclinée sous les angles
    techniques, économiques et humains.}
    \en{Project to transform the IS of an airline company. Benchmarks, studies
    and strategy with IT, economic and humans goals.  Goal definition and the
    trajectory to reach it.}
}

\full{\cventry{2019-2020}{AWS Workshop}{Terraform}
{}{1 \fr{semaine}\en{week}}
{\fr{Cours pratiques pour découvrir l'écosystème AWS}
\en{Practical courses to discover the AWS ecosystem}}}

\full{\cventry{2019}{Serveur Web}{C++}
{\href{https://github.com/tommoulard/sws}{\fr{lien}\en{link} github}}{7 semaines}
{Spider Web Server est un projet pédagogique. Spider est un serveur HTTP}}

\full{\cventry{2019}{Switch Virtuel}{C}
{\href{https://github.com/tommoulard/myvswitch}{\fr{lien}\en{link} github}}{2 semaines}{
Implémentation d'un switch virtuel. Comprends des notions de vlans et de Port
mirroring.}}

\full{\cventry{2018}{Malloc}{C}
{\href{https://github.com/tommoulard/malloc}{\fr{lien}\en{link} github}}{1 weeks}{
Memory allocator without any call to sbrk}}

\full{\cventry{2018-2019}{Google HashCode}{python}
{}{}
{\fr{Compétition mondiale d'implémentation et d'optimisation d'algorithmes
pour résoudre des problèmes NP}
\en{World wide contest to implement and optimize solution for NP problems}}}

\cventry{}{make-my-server}{docker, docker-compose}
{\href{https://github.com/tommoulard/make-my-server}{\fr{lien github}\en{github link}}}{}
{
    \fr{
        Scripts pour manager la configuration de mon server.
    }
    \en{
        Set of script to manage my server configuration.
    }
}

\section{\fr{Compétences}\en{Skills}}

\cventry{\fr{Langages}\en{Languages}}{Go, Python, C, shell}{}
{Java, C++, C\#, OCamL, JS, SQL}{}{}

\cventry{\fr{Outils}\en{Tools}}
{
    Docker, GIT, Kubernetes, Ansible, Terraform, OpenStack, AWS, GCP,
    OVH}{}
{
    Docker-compose, Traefik, TensorFlow, k3s, Django, MongoDB, Neo4j, MySQL, PostgreSQL, VirtualBox,
    Agouti, CircleCI, Apache, Hugo, \textit{R5}, PFSense, Heroku, Unity}{}{}

\cventry{Markup}{{\LaTeX}}{}{HTML, CSS, YAML, toml}{}{}

\cventry{\fr{Multimédia}\en{Multimedia}}{Word, Excel, Access, PowerPoint}{}
{GIMP, Imagemagick, Inkscape}{}{}

\cventry{\fr{Langues}\en{Languages}}
{\fr{Français, Anglais}\en{French, English}}{}
{\fr{Espagnol, Chinois}\en{Spanish, Chinese}}{}
{
    \fr{
        Natale, Bilingue (TOEIC: 945/990), \'Ecrit et parlé, Compréhension
        partielle.
    }
    \en{
        Native, Fluent (TOEIC: 945/990), Writing and Speaking, Low
        understanding
    }
}

\section{\fr{Intérêts}\en{Interests}}
\cvitem{\fr{Technologie}\en{Technology}}
{
    \fr{Fondations Open Source, Robotique, Drones.}
    \en{Open Source fondations; Robotics; drones;}
}

\full{
    \cvitem{\fr{Environnement}\en{Environment}}
    {
        \fr{Impact social et responsable, influence de la technologie.}
        \en{Social impact and responsibility; influence of technology.}
    }
}

\cvitem{Sports}
{
    \fr{Natation, badminton et roller hockey.}
    \en{Swimming; Badminton; Roller hockey.}
}

\end{document}
