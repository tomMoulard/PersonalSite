\documentclass[a4paper]{article}

%% Language and font encodings
\usepackage[english]{babel}
\usepackage[utf8x]{inputenc}
\usepackage[T1]{fontenc}
\usepackage{graphicx}
\usepackage{float}       
%\usepackage{geometry}
\usepackage[left=4cm,right=3cm,top=3cm,bottom=2.5cm]{geometry}
\usepackage{vmargin}
\usepackage{fancyhdr}
\usepackage{epsfig}
\usepackage{eurosym}
\usepackage{url}
\graphicspath{ {figures/} }
\usepackage{array}
\usepackage{listings}
\usepackage{mathtools}

\newenvironment{Times}{\fontfamily{ptm}\selectfont}{}
\fancyhf{}
\pagestyle{fancy}
\fancyfoot[R]{\thepage}
\fancyhead[L]{\leftmark}
\usepackage{setspace}
\setstretch{1.5}

%Python code
\usepackage{listings}
\usepackage{color}

\definecolor{mygreen}{rgb}{0,0.6,0}
\definecolor{mygray}{rgb}{0.5,0.5,0.5}
\definecolor{mymauve}{rgb}{0.58,0,0.82}

\lstset{ %
  backgroundcolor=\color{white},   % choose the background color; you must add \usepackage{color} or \usepackage{xcolor}; should come as last argument
  basicstyle=\footnotesize,        % the size of the fonts that are used for the code
  breakatwhitespace=false,         % sets if automatic breaks should only happen at whitespace
  breaklines=true,                 % sets automatic line breaking
  captionpos=b,                    % sets the caption-position to bottom
  commentstyle=\color{mygreen},    % comment style
  deletekeywords={...},            % if you want to delete keywords from the given language
  escapeinside={\%*}{*)},          % if you want to add LaTeX within your code
  extendedchars=true,              % lets you use non-ASCII characters; for 8-bits encodings only, does not work with UTF-8
  frame=single,                    % adds a frame around the code
  keepspaces=true,                 % keeps spaces in text, useful for keeping indentation of code (possibly needs columns=flexible)
  keywordstyle=\color{blue},       % keyword style
  language=Python,                 % the language of the code
  morekeywords={*,...},            % if you want to add more keywords to the set
  numbers=left,                    % where to put the line-numbers; possible values are (none, left, right)
  numbersep=5pt,                   % how far the line-numbers are from the code
  numberstyle=\tiny\color{mygray}, % the style that is used for the line-numbers
  rulecolor=\color{black},         % if not set, the frame-color may be changed on line-breaks within not-black text (e.g. comments (green here))
  showspaces=false,                % show spaces everywhere adding particular underscores; it overrides 'showstringspaces'
  showstringspaces=false,          % underline spaces within strings only
  showtabs=false,                  % show tabs within strings adding particular underscores
  stepnumber=1,                    % the step between two line-numbers. If it's 1, each line will be numbered
  stringstyle=\color{mymauve},     % string literal style
  tabsize=2,                       % sets default tabsize to 2 spaces
  title=\lstname                   % show the filename of files included with \lstinputlisting; also try caption instead of title
}

\lstdefinestyle{customc}{
  belowcaptionskip=1\baselineskip,
  breaklines=true,
  frame=L,
  xleftmargin=\parindent,
  language=C,
  showstringspaces=false,
  basicstyle=\footnotesize\ttfamily,
  keywordstyle=\bfseries\color{green!40!black},
  commentstyle=\itshape\color{purple!40!black},
  identifierstyle=\color{blue},
  stringstyle=\color{orange},
}

\lstdefinestyle{customasm}{
  belowcaptionskip=1\baselineskip,
  frame=L,
  xleftmargin=\parindent,
  language=[x86masm]Assembler,
  basicstyle=\footnotesize\ttfamily,
  commentstyle=\itshape\color{purple!40!black},
}

\title{\underline{\textbf{\begin{Huge}Network\end{Huge}}}\\\vspace{1cm}
    \begin{huge}
        \textbf{Client / Server communication study}
    \end{huge}\\}
    
\author{
        Tom Moulard (16920041)
        \date{}
        }
        
        
\begin{document}
\maketitle
\begin{center}
\vspace{1cm}
\date{2017 June 12}
\vspace{2cm}\\
\end{center}    
    \thispagestyle{empty}

\newpage{}

\tableofcontents
\thispagestyle{empty}

\newpage
\section{Introduction}

When studying how computers talk to each other, the best way to understand it is to see what they are actually saying to each others.
For this, I created a small Python Program to make two computer communicate, one as a server and the other as a client sending a file to the server.
We are going to see how this program works in details and how the two computers have managed to talk to each other in order to send this file.  

\section{Code}

\subsection{Actual code}
\lstinputlisting[language=Python]{customIperf.py}

\subsection{Explanations}
%This section should explain how the code works

This code initiate the connection between two computers and allow one to send a file to the other.
Indeed, to make this works, we need to have two computers able to interpret python code.
And we need to establish which one is going to be the server and with one the client.
Once it is done, we have to run the interpreter in a terminal like this : \textit{python3 customIperf.py -s} for the server.\\
For the client you should have a little more knowledge : you need to know the IP address of the server. Internal or external, it is up to you to choose depending on how these two computers can talk with each other. And then you can run this : \textit{python3 customIperf.py -c <Server IP address> <file>}. The file can be every file on you computer, it will e stored on the server with the same name.

When running the code, the first thing is to determine what this computer is going to be : \textit{-s} for server, \textit{-c} for client.

\subsubsection*{Server}

Once the computer knows that it is going to be the server it will open the setted port to allow clients to connect to him. To set the port, either the user can set it or the computer can choose a random one.then the server will wait for a client signal to establish the connection.\\
When a client tries to connect to the server the server will gather informations about the client to be able to communicate with him.\\
 Once it is done, the server will wait to receive a number which correspond of the length of the fileName. then it will listen the fileName using it's length and will create a new file with this fileName.\\
After the document has been created, the server will gather all informations going through the network buffer to watch rights packets to add to the file. and when the server sees a EOF signal, it closes the document and closes the connection with the client.
Now another client can connect to the server to send him another file.

\begin{figure}[H]
\centering
\includegraphics[width=0.6\textwidth]{serverSide.png}
\caption{Server side output}
\end{figure}

\subsubsection*{client}

The client is pretty much the opposite of the server.\\
The client is going to try to establish the connection with the ip/port given by the user to the server. If there is nothing, the client will just wait for the server to be started.\\
when the connection is established, it will send the length of the fileName, the fileName and then the document.

\begin{figure}[H]
\centering
\includegraphics[width=0.6\textwidth]{clientSide.png}
\caption{Client side output}
\end{figure}

Also, as you can notice, I made this code in only one file to make it easier to use. %one script to rule them all
it make it easier to transfer between computers and easy to change from client to server or server to client.

In case there is something wrong with the connection, python with throw an exception to tell the user what append to help him correct wrong things. But the only down side is when transferring a file, if something is wrong, my script won't be able to detect it and the output file on the server might be corrupted.

Which lead us to possibles improvement that can be done for this script.
In fact, this is the base of a complicated file transfer like emails, WWW(World Wide Web), FTP(File Transfer Protocol), SSH(Secure SHell), or even streaming data for example.
So there is a lot of ways to make this script better, but it depends on the usage/main characteristics we want it to have like security, effectiveness, ...

\newpage
\section{Interpretation}
%This looks at wireshark screens and explain what the f*** appended

Now that the easy stuff is gone, we can look more in depth on what is really important here: network.

\begin{figure}[H]
\centering
\includegraphics[width=1\textwidth]{WS_-_General_lookup.png}
\caption{Wireshark output}
\end{figure}

This is showing to main display of Wireshark for this file sharing.
We can see the different ports.
Since i am running both the client and the server from my computer they all have the same ip, just the port changes : 10000 for the server and 56158 for the client.\\
Just by looking at it without any detail we can see three distinct part: The Authentication Part (or the Connection Establishment Part), The Data Transfer Part and The Close connection Part. Each follow a natural flow, the authentication first, the data transfer second and then closing the connection. 

\subsection{The Authentication Part}

As we can see in the column Protocol, this connection was established in TCP which means that the authentication part was managed by a three way handshake protocol.

\subsubsection{The (famous) Three Way Handshake}

This protocol is implemented in three phases : the Synchronize(SYN) message, synchronize-Acknowledgment(SYN-ACK) message and the Acknowledgment(ACK) message. This protocol ensure that both client and server is synchronized and ready to communicate.

\subsubsection{The Synchronize Message}

This message is send by the client to ask the server if he is available for a new client to connect to him.
In our case, it is the first packet on the list with the info beginning by "[SYN]" and we can see the port source is the client and the port destination is the server.
We can also notice the the Seq=0 witch means that it is the first packet of this discussion.

\subsubsection{The Synchronize - Acknowledgment Message}

This is the response from the server to the client, as we can see on the source and destination port and on flags : "[SYN - ACK]".
We can observe that the flag Seq is also null showing that it is the server's own sequence but the ACK number is 1 witch means that the server received the first packet.

\subsubsection{The Acknowledgment Message}

This is the final packet send / received for the Handshake. This packet (n°3 here) have an ACK number at 1 (SEQ from the packet n°2 + 1) and the server does not need to reply for this one knowing that both client and server are ready to discuss.

\subsection{The Data Transfer Part}

After the communication has been established and an authentication performed, the real data transfer can begin.
In my code, I specified that the length of the fileName should be less than 10\^512 and therefore the first packet the program send is a packet with a length of 580 : 512 of data ("len=512") and 68 of header. 
For the file i sent, the fileName is equal to "fileToTest" and has a length of 10 characters.
Also, we can note that the ACK number follows the SEQ number of the previous packet.
\\
Then the server replies with another packet, with a SEQ number following the last ACK one, to tell the client he received the packet and he have a god checksum, so there is no data loss.
\\
After this, my program send the fileName encoded in UTF-8 in another packet. And the server replies that he received the packet and everything is right. Since the fileName has a length of 10, the server know he will receive a packet with a data of this length and the "len=10" characteristic can attest this. 
\\
To finish the data sending, the program sends the content of the file. the server will just receive all packet from this client and append them to the file.

\subsection{The Close Connection Part}

To properly finish the connection the client sends a "FIN" packet to tell the server to close the connection and the file.
\\
The server simply respond that he got all packet and that he will close the connection.

\section{Conclusion}

Now we can understand how theses communication occurred to transmit a file between two computers using a TCP connection. 

\newpage
\listoffigures
\end{document}